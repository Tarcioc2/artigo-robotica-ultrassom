\chapter{Conclusão}
\label{chap:conc}

O uso de sensores ultra-sônicos vêm crescendo com o passar do tempo, tornando-se uma boa opção quando se necessita de uma grande precisão nos dados de posição indoor obtidos, porém apesar desse benefício o uso dessa tecnologia em ambientes muito extensos podem causar problemas na escala de precisão, visto que à quantidade de pontos cegos aumentará. Para a correção deste problema, diversos novos pontos de antena deverão ser posicionados durante a configuração inicial. Além disso, apesar dos baixos custos iniciais de instalação, antes de optar por esse tipo de sensores na construção do seu sistema de localização indoor, deve-se levar em consideração que os mesmos têm uma sensibilidade elevada a barulhos e ondas de choque externas, sendo necessário o tratamento dos ambientes e/ou correções nos cálculos das equações envolvidas no sistema para sobrepor esses problemas.




