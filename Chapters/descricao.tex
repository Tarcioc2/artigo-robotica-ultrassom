\chapter{Descrição da Tecnologia}
\label{chap:fundteor}



A tecnologia Ultrassom é um som emitido em frequência superior a que os seres humanos podem escutar, acima de 20 kHz. O ultrassom é amplamente utilizado para diversas funções diferentes, tais como a Ultrassonografia para a medicina, tanto veterinária como humana, é utilizada também em testes não destrutivos, à fim de identificar problemas estruturais ou falhas em produtos no geral, e até mesmo como detector de objetos e medir distâncias, ajudando, por exemplo, no posicionamento de robôs móveis em ambientes controlados e não muito amplos.\\
\\
A utilização da tecnologia Ultrassom para posicionamento indoor tem como principal vantagem a precisão no cálculo de posicionamento, que é promovida através de uma aplicação técnica chamada Time of Sight (TOS), que é baseada no tempo de envio e recebimento de sinais aos sensores, gerando informações em três dimensões. \\
\\
Inicialmente, será necessário realizar o mapeamento da área de cobertura, definindo os pontos de acesso e o posicionamento das antenas que deverão ter as suas coordenadas configuradas de forma manual para à base de dados, essa é uma etapa crucial para a configuração do sistema. Com a finalidade de diminuir interferências que possam ser causadas pelo ambiente, é recomendado que sejam distribuídas muitas antenas de maneira uniforme.\\
