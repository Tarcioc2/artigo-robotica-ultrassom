\chapter{Introdução}
\label{chap:intro}


Um problema crucial para a robótica móvel na execução de tarefas em longo prazo é determinar e acompanhar a posição do robô em relação ao espaço. Devido a isso, o uso de sensores embarcados se torna cada vez mais comum a soluções envolvendo robótica móvel.\\
\\
Dentre a vasta gama de sensores e possibilidades utilizados nessas soluções, a escolha por sensores ultrassônicos vem sendo bastante difundida devido ao seu baixo custo, baixo tempo de processamento e infraestrutura de fácil montagem. Seu estudo relacionado à localização indoor pode ser subdividido em duas categorias: Sendo uma delas o mapeamento completo do ambiente indoor quando a posição do robô móvel é determinada, e a outra a determinação da localização do robô móvel através da informação sensorial sobre o ambiente em que o mesmo se encontra.\\
\\
No entanto, apesar das vantagens mencionadas acima, um sistema de localização indoor baseado em sensores ultrassom possui diversos problemas para atingir valores apurados de precisão, pois os mesmos têm uma sensibilidade elevada a barulhos e ondas de choque externas, sendo necessário o tratamento dos ambientes e cálculos envolvidos no sistema para sobrepor esses problemas.\\
\\
O presente trabalho descreve nas seções posteriores de forma sucinta essa tecnologia, exemplificando sua utilização na robótica móvel a fim de ambientar o leitor quanto ao assunto.\\


\ \\
