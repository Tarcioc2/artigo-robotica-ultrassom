% a-project.tex, v-1.0.3 marcoreis baseado no
% abntex2-modelo-trabalho-academico.tex, v-1.9.7 laurocesar
% Copyright 2012-2018 by abnTeX2 group at http://www.abntex.net.br/ 
% 
% This work consists of the files ........
% 
% -----------------------------------------------------------------------------
% Modelo para desenvolvimento de documentação de projetos acadêmicos
% (tese de doutorado, dissertação de mestrado e trabalhos de monografias em geral) 
% em conformidade com ABNT NBR 14724:2011: Informação e documentação. 
% -----------------------------------------------------------------------------
% Opções para a documentação
%
% Fancy page headings 
%\documentclass[fancyheadings, subook]{Classes/a-prj}
%\documentclass[fancyheadings, sureport]{Classes/a-prj}
%
% Fancy chapters and sections headings 
%\documentclass[fancychapter, subook]{Classes/a-prj}
%\documentclass[fancychapter, sureport]{Classes/a-prj}
%
% Fancy page , chapters and sections headings
%\documentclass[fancyheadings, fancychapter, subook]{Classes/a-prj}
\documentclass[fancyheadings, fancychapter, sureport]{Classes/a-prj}
%
% -----------------------------------------------------------------------------
% Alguns comandos para a fancy page headings)
%
% Page header line width
%\footlinewidth{value}
%
% Page footer line width
%\headlinewidth{value}
%
% Page header and footer line width
%\headingslinewidth{value}
%
% Page header and footer lines without text
%\headingslinesonly
%
% The default line width is 0.3pt.
% Set the value to 0pt to remove the page header and/or footer line
%
% -------------------------------------------------------------------------------
% Formato de figuras suportado
% -------------------------------------------------------------------------------
% O formato das figuras depende da forma como o arquivo de saída é gerado.
% As figuras inseridas na pasta Figures serão automaticamente reconhecidas sem
% a necessidade de inserir a extensão do arquivo.
%
% O pdfLaTEX (PDF) suporta figuras com as extensões: pdf, jpg, png e mps.
%
% -------------------------------------------------------------------------------
% Árvore do diretório a-project.tex
%  Diretório
%       \Classes        (requerido)
%       \Figures        (requerido) --------------------------------->
%       \Figures\PDF    (optional)
%       \Figures\JPG    (optional) Figures located within these
%       \Figures\PNG    (optional) folders are searched automatically
%       \Figures\MPS    (optional)  by the a-prj class.
%       \Figures\EPS    (optional)
%       \Figures\PS     (optional) <--------------------------------
%       \Tables         (requerido)
%       \Others         (requerido)
%       \Chapters       (requerido)
%       \Appendices     (optional)
%       \References     (requerido)
%
% -------------------------------------------------------------------------------
% PDF File resumo
\ifpdf
    \hypersetup{
    	backref,
        colorlinks  = true,
        pdftitle    = Modelo de documentação,
        pdfauthor   = {Marco Reis, marco.a.reis@gmail.com},
        pdfsubject  = Mestre em Engenharia,
        pdfcreator  = Subtitulo,
        pdfproducer = PDFLatex,
        pdfkeywords = {documentação, latex, dissertação, tese}}
 \fi
%
% -------------------------------------------------------------------------------
% Relação de pacotes opcionais utilizados
\usepackage[utf8]{inputenc}
\usepackage[brazil]{babel}
\usepackage{longtable}
\usepackage{dcolumn}
\usepackage{multirow}
\usepackage{lscape}
%\usepackage{graphicx}
\usepackage{rotating}
%\usepackage{float,subfigure}
%\usepackage{graphicx, subfigure}
\usepackage{cite}
\usepackage[left=3cm,top=3cm,right=2cm,bottom=2cm]{geometry}
\usepackage[alf]{abntex2cite}
\usepackage{ifpdf}
\usepackage{shadow}
\usepackage{wrapfig}
\usepackage[normalem]{ulem}
\usepackage{makeidx}
\usepackage{yfonts}
\usepackage{algorithm}
\usepackage{algorithmic}
\usepackage{lmodern}
\usepackage[T1]{fontenc}
\usepackage{indentfirst}
\usepackage{color}
\usepackage{microtype}
\usepackage{lipsum}
\usepackage{caption}
\usepackage{subcaption}
%
\makeindex 
\setlength{\LTcapwidth}{\textwidth}
%
\newtheorem{theorem}{Teorema}
\newtheorem{definition}[theorem]{Definição}
%
% -------------------------------------------------------------------------------
% Configurações do pacote backref
\renewcommand{\backrefpagesname}{Citado na(s) página(s):~}
% Texto padrão antes do número das páginas
\renewcommand{\backref}{}
% Define os textos da citação
\renewcommand*{\backrefalt}[4]{
	\ifcase #1 %
		Nenhuma citação no texto.%
	\or
		Citado na página #2.%
	\else
		Citado #1 vezes nas páginas #2.%
	\fi
}
% 
% -------------------------------------------------------------------------------
% Início do documento raiz
\begin{document}
% Definição do título da página
    \university{Centro Universitário SENAI CIMATEC}
	%\faculty{Programa de...}
	%\school{Escola de...}
% 
    %\course{Engenharia Elétrica}
    \typework{Fundamentos de Robótica Móvel}
% 
	%\course{Mestrado em Modelagem Computacional e Tecnologia Industrial}
	%\typework{Disserta\c{c}\~ao de mestrado}
	%\typework{Exame de Qualificação de Mestrado}
% 
	%\course{Engenharia Elétrica}
	%\typework{Tese de doutorado}
	%\typework{Exame de Qualificação de doutorado}
%
% -------------------------------------------------------------------------------
% Informações gerais
    \thesistitle{Estudo Direcionado Para o Posicionamento Indoor de Robôs Móveis Utilizando Sensores Ultrassom}
    \hidevolume
    \thesisvolume{Volume 1 of 1}
    \thesisauthor{Gabriel Luiz}
    \thesisauthorr{Lucas Cassimiro}
    \thesisauthorrr{Tárcio Carvalho}
    %\thesisadvisor{Prof. Marco Reis, M.Eng.}
    %\hidecoadvisor
    %\thesiscoadvisor{Marco Reis}
    \thesisdegreetitle{Bacharel em Engenharia}
    \thesismonthyear{Agosto de 2020}
% 
    \maketitlepage
%

% Resumo/abstract, sumário e siglas
    \begin{romanpagenumbers}
        \begin{thesisresumo}
No presente trabalho é desenvolvido um estudo sobre a tecnologia de localização utilizando sensores ultrassônicos, trazendo uma sucinta explicação do seu funcionamento e exemplificando sua aplicação na odometria e localização indoor para a robótica móvel.

\ \\
\textbf{Keywords}: Robótica, Ultrassom, Posicionamento, Indoor, Odometria
\end{thesisresumo}

        \begin{thesisabastract}
In the present work, a study on location technology using ultrasonic sensors is developed, providing a brief explanation of its operation and exemplifying its application in odometry and indoor location for mobile robotics.

\ \\

% use de tr�s a cinco palavras-chave

\textbf{Keywords}: Robotics, Ultrasonic, Positioning, Indoor, Odometry

\end{thesisabastract}

        % Make list of contents, tables and figures
        %Include other required section
        %\begin{thesisabbreviations}
\begin{footnotesize}
\begin{longtable}[l]{p{2cm}l}
  tprax   \dotfill & \thefaculty \\
  WWW       \dotfill &  World Wide Web \\
\end{longtable}
\end{footnotesize}
\end{thesisabbreviations}

        %\begin{thesissymbols}
\begin{footnotesize}
\begin{longtable}[l]{p{2cm}l}
  $\partial$   \dotfill  & Bla bla bla \\
  $\prod$       \dotfill & ble ble ble \\
  $\partial$   \dotfill  & Bla bla bla \\
  $\prod$       \dotfill & ble ble ble \\
  $\partial$   \dotfill  & Bla bla bla \\
  $\prod$       \dotfill & ble ble ble \\
  $\partial$   \dotfill  & Bla bla bla \\
  $\prod$       \dotfill & ble ble ble \\
  $\partial$   \dotfill  & Bla bla bla \\
  $\prod$       \dotfill & ble ble ble \\
  $\partial$   \dotfill  & Bla bla bla \\
  $\prod$       \dotfill & ble ble ble \\
  $\partial$   \dotfill  & Bla bla bla \\
  $\prod$       \dotfill & ble ble ble \\
  $\partial$   \dotfill  & Bla bla bla \\
  $\prod$       \dotfill & ble ble ble \\
  $\partial$   \dotfill  & Bla bla bla \\
  $\prod$       \dotfill & ble ble ble \\
  $\partial$   \dotfill  & Bla bla bla \\
  $\prod$       \dotfill & ble ble ble \\
  $\partial$   \dotfill  & Bla bla bla \\
  $\prod$       \dotfill & ble ble ble \\
  $\partial$   \dotfill  & Bla bla bla \\
  $\prod$       \dotfill & ble ble ble \\
  $\partial$   \dotfill  & Bla bla bla \\
  $\prod$       \dotfill & ble ble ble \\
  $\partial$   \dotfill  & Bla bla bla \\
  $\prod$       \dotfill & ble ble ble \\
  $\partial$   \dotfill  & Bla bla bla \\
  $\prod$       \dotfill & ble ble ble \\
  $\partial$   \dotfill  & Bla bla bla \\
  $\prod$       \dotfill & ble ble ble \\
  $\partial$   \dotfill  & Bla bla bla \\
  $\prod$       \dotfill & ble ble ble \\
  $\partial$   \dotfill  & Bla bla bla \\
  $\prod$       \dotfill & ble ble ble \\
  $\partial$   \dotfill  & Bla bla bla \\
  $\prod$       \dotfill & ble ble ble \\          
\end{longtable}
\end{footnotesize}
\end{thesissymbols}

        %Switch the page numbering back to the default format
    \end{romanpagenumbers}
%
% ---------------------------------------------------------------------------
% Include thesis chapters
    \parskip=\baselineskip
    \chapter{Introdução}
\label{chap:intro}


Um problema crucial para a robótica móvel na execução de tarefas em longo prazo é determinar e acompanhar a posição do robô em relação ao espaço. Devido a isso, o uso de sensores embarcados se torna cada vez mais comum a soluções envolvendo robótica móvel.\\
\\
Dentre a vasta gama de sensores e possibilidades utilizados nessas soluções, a escolha por sensores ultrassônicos vem sendo bastante difundida devido ao seu baixo custo, baixo tempo de processamento e infraestrutura de fácil montagem. Seu estudo relacionado à localização indoor pode ser subdividido em duas categorias: Sendo uma delas o mapeamento completo do ambiente indoor quando a posição do robô móvel é determinada, e a outra a determinação da localização do robô móvel através da informação sensorial sobre o ambiente em que o mesmo se encontra.\\
\\
No entanto, apesar das vantagens mencionadas acima, um sistema de localização indoor baseado em sensores ultrassom possui diversos problemas para atingir valores apurados de precisão, pois os mesmos têm uma sensibilidade elevada a barulhos e ondas de choque externas, sendo necessário o tratamento dos ambientes e cálculos envolvidos no sistema para sobrepor esses problemas.\\
\\
O presente trabalho descreve nas seções posteriores de forma sucinta essa tecnologia, exemplificando sua utilização na robótica móvel a fim de ambientar o leitor quanto ao assunto.\\


\ \\

    \chapter{Descrição da Tecnologia}
\label{chap:fundteor}



A tecnologia Ultrassom é um som emitido em frequência superior a que os seres humanos podem escutar, acima de 20 kHz. O ultrassom é amplamente utilizado para diversas funções diferentes, tais como a Ultrassonografia para a medicina, tanto veterinária como humana, é utilizada também em testes não destrutivos, à fim de identificar problemas estruturais ou falhas em produtos no geral, e até mesmo como detector de objetos e medir distâncias, ajudando, por exemplo, no posicionamento de robôs móveis em ambientes controlados e não muito amplos.\\
\\
A utilização da tecnologia Ultrassom para posicionamento indoor tem como principal vantagem a precisão no cálculo de posicionamento, que é promovida através de uma aplicação técnica chamada Time of Sight (TOS), que é baseada no tempo de envio e recebimento de sinais aos sensores, gerando informações em três dimensões. \\
\\
Inicialmente, será necessário realizar o mapeamento da área de cobertura, definindo os pontos de acesso e o posicionamento das antenas que deverão ter as suas coordenadas configuradas de forma manual para à base de dados, essa é uma etapa crucial para a configuração do sistema. Com a finalidade de diminuir interferências que possam ser causadas pelo ambiente, é recomendado que sejam distribuídas muitas antenas de maneira uniforme.\\

    \chapter{Aplicação na Robótica}
\label{chap:mat}

A Universidade Federal do Espírito Santo (UFES)  construiu um robô móvel de tração diferencial. O robô foi batizado como Brutus, sua estrutura básica possui  uma plataforma circular e quatro rodas, duas dessas rodas são rodas livres(sem nenhum tipo de tração) e as outras duas são  tracionadas por motores  de corrente contínua.  Esse robô foi construído com a finalidade de prever e evitar colisões enquanto se desloca em algum ambiente de terreno regular onde o seu sistema não possui um conhecimento prévio sobre os obstáculos e limitações intrínsecas nesse ambiente.\\
\\
A nível de hardware o robô conta com um microcontrolador MC68332 que é responsável por controlar sensores, lógicas comportamentais e os atuadores. Já a nível de software o Brutus dispõe de um sistema em C++, esse sistema recebe informações dos sensores, com base nessas informações é definido uma saída para seus atuadores, no caso os motores DC conectados na placa microcontroladora.\\
\\
O fluxo do sistema se inicia quando o sensor ultrassônico emite ondas sonoras ciclicamente no ambiente, com base nisso é medido o tempo que essas ondas colidem em um objeto e geram um eco que é recebido pelo sensor e convertido em um sinal elétrico, tornando assim possível a distância dos objetos.  O sinal elétrico é enviado para a placa que faz a conversão do sinal elétrico e através de lógica são definidos os valores de saída para os motores DC, seja para mudar a orientação do robô ou a velocidade.\\



    \chapter{Conclusão}
\label{chap:conc}

O uso de sensores ultra-sônicos vêm crescendo com o passar do tempo, tornando-se uma boa opção quando se necessita de uma grande precisão nos dados de posição indoor obtidos, porém apesar desse benefício o uso dessa tecnologia em ambientes muito extensos podem causar problemas na escala de precisão, visto que à quantidade de pontos cegos aumentará. Para a correção deste problema, diversos novos pontos de antena deverão ser posicionados durante a configuração inicial. Além disso, apesar dos baixos custos iniciais de instalação, antes de optar por esse tipo de sensores na construção do seu sistema de localização indoor, deve-se levar em consideração que os mesmos têm uma sensibilidade elevada a barulhos e ondas de choque externas, sendo necessário o tratamento dos ambientes e/ou correções nos cálculos das equações envolvidas no sistema para sobrepor esses problemas.





    % include more chapters ...
%
% ----------------------------------------------------------------------------
% Include thesis appendices
%
% ----------------------------------------------------------------------------
% Configurar as referencias bibliograficas
	\renewcommand\bibname{Referências}
    \addcontentsline{toc}{chapter}{Referências}
    \bibliography{References/referencias}
    [1]H. R. Beom and H. S. Cho (1995). Mobile robot localization using a single rotating sonar and two passive cylindrical beacons. Robotica, 13, pp 243-252 doi:10.1017/S026357470001777X

    [2] Joseph Azeta et al 2019 IOP Conf. Ser.: Mater. Sci. Eng. 707 012012

    [3] NASCIMENTO, Rafaella Cristiane Alves. Localização de Robôs Móveis em Ambientes Fechados Utilizando Câmeras Montadas no Teto. Natal, 2014. 
%
% ----------------------------------------------------------------------------
% Finishing him
    \include{Others/ultimafolha}
\end{document}
%
% -------------------------------------------------------------------------------
% Aqui termina a formatação para o documento.
% In God We Trust. All Other Bring Data. 
%
% -------------------------------------------------------------------------------